% Options for packages loaded elsewhere
\PassOptionsToPackage{unicode}{hyperref}
\PassOptionsToPackage{hyphens}{url}
%
\documentclass[
]{article}
\usepackage{amsmath,amssymb}
\usepackage{lmodern}
\usepackage{iftex}
\ifPDFTeX
  \usepackage[T1]{fontenc}
  \usepackage[utf8]{inputenc}
  \usepackage{textcomp} % provide euro and other symbols
\else % if luatex or xetex
  \usepackage{unicode-math}
  \defaultfontfeatures{Scale=MatchLowercase}
  \defaultfontfeatures[\rmfamily]{Ligatures=TeX,Scale=1}
\fi
% Use upquote if available, for straight quotes in verbatim environments
\IfFileExists{upquote.sty}{\usepackage{upquote}}{}
\IfFileExists{microtype.sty}{% use microtype if available
  \usepackage[]{microtype}
  \UseMicrotypeSet[protrusion]{basicmath} % disable protrusion for tt fonts
}{}
\makeatletter
\@ifundefined{KOMAClassName}{% if non-KOMA class
  \IfFileExists{parskip.sty}{%
    \usepackage{parskip}
  }{% else
    \setlength{\parindent}{0pt}
    \setlength{\parskip}{6pt plus 2pt minus 1pt}}
}{% if KOMA class
  \KOMAoptions{parskip=half}}
\makeatother
\usepackage{xcolor}
\IfFileExists{xurl.sty}{\usepackage{xurl}}{} % add URL line breaks if available
\IfFileExists{bookmark.sty}{\usepackage{bookmark}}{\usepackage{hyperref}}
\hypersetup{
  hidelinks,
  pdfcreator={LaTeX via pandoc}}
\urlstyle{same} % disable monospaced font for URLs
\usepackage[margin=1in]{geometry}
\usepackage{graphicx}
\makeatletter
\def\maxwidth{\ifdim\Gin@nat@width>\linewidth\linewidth\else\Gin@nat@width\fi}
\def\maxheight{\ifdim\Gin@nat@height>\textheight\textheight\else\Gin@nat@height\fi}
\makeatother
% Scale images if necessary, so that they will not overflow the page
% margins by default, and it is still possible to overwrite the defaults
% using explicit options in \includegraphics[width, height, ...]{}
\setkeys{Gin}{width=\maxwidth,height=\maxheight,keepaspectratio}
% Set default figure placement to htbp
\makeatletter
\def\fps@figure{htbp}
\makeatother
\setlength{\emergencystretch}{3em} % prevent overfull lines
\providecommand{\tightlist}{%
  \setlength{\itemsep}{0pt}\setlength{\parskip}{0pt}}
\setcounter{secnumdepth}{-\maxdimen} % remove section numbering
\usepackage{float}
\usepackage{sectsty}
\ifLuaTeX
  \usepackage{selnolig}  % disable illegal ligatures
\fi

\author{}
\date{\vspace{-2.5em}}

\begin{document}

\hypertarget{introduction}{%
\section{Introduction}\label{introduction}}

It is important in sports to gain insight and knowledge of the physical
demands from training and matches to improve performance and prevent
injuries. Some of the most frequently reported metrics in football are:
distance covered, distance in velocity zones and number of acc- \&
decelerations \citep{Aughey2011}. These metrics are most commonly being
measured with the use of Global Positioning Systems (GPS), which have
seen an exponentially growth in outdoor team-sports. GPS is validated
for use in outdoor team sports with a sampling frequency of minimum 10
Hz \citep{Scott2016, Johnston2014}. Furthermore, \citet{Petersen2009}
also found that the reliability of the GPS was independent on the time
of day. Nonetheless, these metrics mostly focus on linear movements and
does not take into account the load from small and energy demanding
non-linear movements like change of directions, jumps or the amplitude
of an acceleration or deceleration. These are all relevant aspects for
team coaches since it provides more details on the intensity of a
training session or a game.

Football has over the last decade experienced an increase in the
quantity of short bouts of activity, compared to total distance
travelled \citep{Barnes2014} and some studies have shown upwards of
eightfold more occurrences of accelerations than sprints
\citep{Bradley2010, Varley2012}. Accelerations also has a greater
metabolic demand than steady state running
\citep{Prampero2005, Osgnach2010, Stevens2015}, therefore a huge part of
the physical demand is never accounted for with the use of the
conventional metrics from GPS data. Thus, it is important for the
practitioners to monitor the acc- \& decceleration profile of the
players. Though, GPS has shown poor validity with accelerations above
\(4\: m\cdot s^{-2}\) \citep{Akenhead2014}.

Another approach is to use Inertial Measurement Units (IMU), since these
are now miniaturized and non obstructive \citep{Zheng2014}. IMUs are
experiencing an increase in popularity and have been successfully used
in both outdoor and indoor team sports, to measure the physical demands
\citep{GomezCarmona2020, Cust2018}. The IMUs in Catapult Sports'
tracking device, Minimax S4 and OptimEye S5, has shown to have an
acceptable level of validity and reliability
\citep{Wundersitz2015, Luteberget2018, Nicolella2018}. One of the most
commonly used metrics in football from an IMU, is the magnitude of
accelerations (square root of the summed squares of the three axis). The
magnitude of acceleration is the instantaneous rate of change in
acceleration across all three axis. This has been given names like
\(PlayerLoad^{TM}\) (Catapult Sports) and \(Body Load\) (GPSports).
Where \(PlayerLoad^{TM}\) has shown acceptable reliability during
Australian Football matches both within and between the used devices
(MinimaxX 2.0, Catapult Sports) \citep{Boyd2011}.

Based on the accelerometer data from the IMU, and therefore also the
magnitude of accelerations, Catapult Sports have developed a sport
specific movement algorithm (Football Movement profile (FMP)). The FMP
is an attempt to create a deeper insight into the demands of football.
The algorithm classify movements in six categories depend on intensity
and if it is a linear or a multi directional movement;

\begin{itemize}
\tightlist
\item
  \textbf{Very Low intensity} - Standing-like movements
\item
  \textbf{Low intensity} - Walking-like movements
\item
  \textbf{Running medium intensity} - Steady jogging / running (linear
  locomotion)
\item
  \textbf{Running high intensity} - Steady high-speed running (linear
  locomotion)
\item
  \textbf{Dynamic medium intensity} - Mid-intensity changes of direction
  and accelerations (non-linear locomotion)
\item
  \textbf{Dynamic high intensity} - High-intensity changes of direction
  and accelerations (non-linear locomotion)
\end{itemize}

These are all derived at 1 sec epochs (1 Hz). By utilizing linear
acceleration data, it is possible to get a deeper and more relevant
insight into the loading characteristics in players, compared to linear
velocity data. Though, it is possible to derive linear acceleration from
linear velocity data calculating its derivative \$ v(t) = \frac{dr}{dt}
\$. However, working with acceleration data from IMUs can be cumbersome
and the alternative would be to use motion capture analysis, but these
are highly time-consuming \citep{Petersen2009}. Therefore, by creating
acceleration profiles, FMP, it is now easily accessible for the
practitioners to use to monitor player demands in both training and
matches. This is especially important for elite football teams who have
the means to buy and utilize the collected data, as it assists in the
insight of the individual player load. This also makes the data usable
during training or matches, which would not be the case for a motion
capture analysis approach.

By using FMP, there is also a potential to create ``fingerprints'' for
different drills. As it is expected that different drills consist of
different distributions of the six locomotion categories and can be used
to identify drill types. This can potentially assist coaches in planning
training routines as they will now have an insight to the loading
amplitude of each drill. Though, this requires FMP to be properly
validated. Therefore, a part of this study will be allocated to validate
the locomotion part (linear or non-linear movement) of the FMP but not
the intensity. Intensity will not be validated as there are currently no
standard to compare it to. Velocity intensity zones are broadly used but
these zones have been arbitrary chosen without any robust theoretical
framework behind it \citep{Malone2017}. Furthermore, velocity from GPS
data are not euclidean vectors as they do not have a direction, which
accelerations from IMUs does. Consequently, if an athlete performs a
hard deceleration during a change of direction the velocity would fall
to a ``low intensity'' zone, even though very high amplitude of
accelerations are being executed. For linear running locomotion there
are also great differences in running technique which affects running
economy and performance. Vertical oscillation, range in the transverse
rotation, range of horizontal velocity during ground contact and more,
are all factors that impacts the running economy \citep{Folland2017}
that the GPS can not measure. To validate the locomotion of FMP,
different simplified drills can be executed to avoid inter-rater
reliability issues \citep{Reinking2018}.

Therefore, the aim of this study was twofold. 1) Validate the locomotion
predictions of FMP during controlled and simplified drills. 2) Create an
algorithm based on FMP that can predict different drill types.

\end{document}
